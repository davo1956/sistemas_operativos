\documentclass[12pt]{article}

\usepackage[spanish]{babel}
\usepackage[utf8]{inputenc}
\usepackage{graphicx}
\usepackage{geometry}
\usepackage{xcolor}
\usepackage{fancyhdr}
\usepackage{lastpage}
\usepackage{pdfpages}
\usepackage{listings}

\geometry{top=25mm,left=15mm,right=15mm,a4paper}

\pagestyle{fancy}
\fancyhf{}
\lhead{Sistemas Operativos}
\cfoot{Página \thepage\ de \pageref{LastPage}}

\graphicspath{./}

\begin{document}
\includepdf{Portada.pdf}
{\color{blue} \section*{Tarea 4.}}

{\color{blue} \subsection*{Instrucciones.}}
\vspace{0.5em} 

Lee con atención las preguntas y contesta lo correspondiente. La tarea se entregará por vía classroom
en un archivo pdf que debe tener el nombre completo y número de tarea, ya sea en una portada o en el encabezado.
\textbf{La tarea se entregará en equipos a lo más de dos personas.}\\

{\color{blue} \subsection*{Ejercicios}}
\vspace{0.5em}

\begin{enumerate}
    \item Describe a detalle la inversión de prioridad de procesos.
    \vspace{2mm}

    \textbf{RESPUESTA}

    \item Dada la inversión de prioridad de procesos. ¿Qué inconvenientes tiene? ¿Cómo se podría solucionar?
    \vspace{2mm}
    
    \textbf{RESPUESTA}
    
    \item ¿Qué es la memoria caché? ¿Para qué sirve?.
    \vspace{2mm}

    \textbf{RESPUESTA}

    \item ¿Qué es el "base register" y el "limit register" respecto al espacio de memoria de cada proceso?
    \vspace{2mm}

    \textbf{RESPUESTA}

    \item Describe con detalle qué es swapping y por qué es necesario. 
    \vspace{2mm}

    \textbf{RESPUESTA}

    \item Explica qué es segmentación, describe pros y contras.
    \vspace{2mm}

    \textbf{RESPUESTA}

    \item ¿Qué es la tabla de segmentos?  
    \vspace{2mm}

    \textbf{RESPUESTA}

    \item Explica qué es la MMU
    \vspace{2mm}

    \textbf{RESPUESTA}
    
    \item Explica qué es paginación, describe pros y contras.
    \vspace{2mm}

    \textbf{RESPUESTA}

    \item ¿Qué es compartición?
    \vspace{2mm}
    
    \textbf{RESPUESTA}

    \item ¿De qué manera determinas a que página pertenece una dirección virtual en paginación?
    \vspace{0mm}
    \textbf{RESPUESTA}

    \item ¿Qué es un frame y qué es un page? ¿Cúal es su relación?
    \vspace{2mm}
    
    \textbf{RESPUESTA}

    \item Describe la paginación de varios niveles y que beneficios tiene en comparación a la paginación normal.
    \vspace{2mm}
    
    \textbf{RESPUESTA}

    \item Explica la técnica que se usa para mitigar los efectos negativos de usar múltiples niveles en la técnica de paginación.
    \vspace{2mm}
    
    \textbf{RESPUESTA}

    \item ¿Qué es el TLB (Translation Lookside Buffer)?
    \vspace{2mm}
    
    \textbf{RESPUESTA}

    \item Dada la siguiente cadena de referencia: $4,7,5,3,2,3,5,3,7,6,0,1,4,0,7,1,6.$ Realiza el algoritmo de FIFO Page Replacement con 4 Frames
    \vspace{2mm}
    
    \textbf{RESPUESTA}

    \item Dada la siguiente cadena de referencia: $4,7,5,3,2,3,5,3,7,6,0,1,4,0,7,1,6.$ Realiza el algoritmo de Optimal Page Replacement con 4 Frames.
    \vspace{2mm}
    
    \textbf{RESPUESTA}

\end{enumerate}

\end{document}